
% Default to the notebook output style

    


% Inherit from the specified cell style.




    
\documentclass[11pt]{article}

    
    
    \usepackage[T1]{fontenc}
    % Nicer default font (+ math font) than Computer Modern for most use cases
    \usepackage{mathpazo}

    % Basic figure setup, for now with no caption control since it's done
    % automatically by Pandoc (which extracts ![](path) syntax from Markdown).
    \usepackage{graphicx}
    % We will generate all images so they have a width \maxwidth. This means
    % that they will get their normal width if they fit onto the page, but
    % are scaled down if they would overflow the margins.
    \makeatletter
    \def\maxwidth{\ifdim\Gin@nat@width>\linewidth\linewidth
    \else\Gin@nat@width\fi}
    \makeatother
    \let\Oldincludegraphics\includegraphics
    % Set max figure width to be 80% of text width, for now hardcoded.
    \renewcommand{\includegraphics}[1]{\Oldincludegraphics[width=.8\maxwidth]{#1}}
    % Ensure that by default, figures have no caption (until we provide a
    % proper Figure object with a Caption API and a way to capture that
    % in the conversion process - todo).
    \usepackage{caption}
    \DeclareCaptionLabelFormat{nolabel}{}
    \captionsetup{labelformat=nolabel}

    \usepackage{adjustbox} % Used to constrain images to a maximum size 
    \usepackage{xcolor} % Allow colors to be defined
    \usepackage{enumerate} % Needed for markdown enumerations to work
    \usepackage{geometry} % Used to adjust the document margins
    \usepackage{amsmath} % Equations
    \usepackage{amssymb} % Equations
    \usepackage{textcomp} % defines textquotesingle
    % Hack from http://tex.stackexchange.com/a/47451/13684:
    \AtBeginDocument{%
        \def\PYZsq{\textquotesingle}% Upright quotes in Pygmentized code
    }
    \usepackage{upquote} % Upright quotes for verbatim code
    \usepackage{eurosym} % defines \euro
    \usepackage[mathletters]{ucs} % Extended unicode (utf-8) support
    \usepackage[utf8x]{inputenc} % Allow utf-8 characters in the tex document
    \usepackage{fancyvrb} % verbatim replacement that allows latex
    \usepackage{grffile} % extends the file name processing of package graphics 
                         % to support a larger range 
    % The hyperref package gives us a pdf with properly built
    % internal navigation ('pdf bookmarks' for the table of contents,
    % internal cross-reference links, web links for URLs, etc.)
    \usepackage{hyperref}
    \usepackage{longtable} % longtable support required by pandoc >1.10
    \usepackage{booktabs}  % table support for pandoc > 1.12.2
    \usepackage[inline]{enumitem} % IRkernel/repr support (it uses the enumerate* environment)
    \usepackage[normalem]{ulem} % ulem is needed to support strikethroughs (\sout)
                                % normalem makes italics be italics, not underlines
    

    
    
    % Colors for the hyperref package
    \definecolor{urlcolor}{rgb}{0,.145,.698}
    \definecolor{linkcolor}{rgb}{.71,0.21,0.01}
    \definecolor{citecolor}{rgb}{.12,.54,.11}

    % ANSI colors
    \definecolor{ansi-black}{HTML}{3E424D}
    \definecolor{ansi-black-intense}{HTML}{282C36}
    \definecolor{ansi-red}{HTML}{E75C58}
    \definecolor{ansi-red-intense}{HTML}{B22B31}
    \definecolor{ansi-green}{HTML}{00A250}
    \definecolor{ansi-green-intense}{HTML}{007427}
    \definecolor{ansi-yellow}{HTML}{DDB62B}
    \definecolor{ansi-yellow-intense}{HTML}{B27D12}
    \definecolor{ansi-blue}{HTML}{208FFB}
    \definecolor{ansi-blue-intense}{HTML}{0065CA}
    \definecolor{ansi-magenta}{HTML}{D160C4}
    \definecolor{ansi-magenta-intense}{HTML}{A03196}
    \definecolor{ansi-cyan}{HTML}{60C6C8}
    \definecolor{ansi-cyan-intense}{HTML}{258F8F}
    \definecolor{ansi-white}{HTML}{C5C1B4}
    \definecolor{ansi-white-intense}{HTML}{A1A6B2}

    % commands and environments needed by pandoc snippets
    % extracted from the output of `pandoc -s`
    \providecommand{\tightlist}{%
      \setlength{\itemsep}{0pt}\setlength{\parskip}{0pt}}
    \DefineVerbatimEnvironment{Highlighting}{Verbatim}{commandchars=\\\{\}}
    % Add ',fontsize=\small' for more characters per line
    \newenvironment{Shaded}{}{}
    \newcommand{\KeywordTok}[1]{\textcolor[rgb]{0.00,0.44,0.13}{\textbf{{#1}}}}
    \newcommand{\DataTypeTok}[1]{\textcolor[rgb]{0.56,0.13,0.00}{{#1}}}
    \newcommand{\DecValTok}[1]{\textcolor[rgb]{0.25,0.63,0.44}{{#1}}}
    \newcommand{\BaseNTok}[1]{\textcolor[rgb]{0.25,0.63,0.44}{{#1}}}
    \newcommand{\FloatTok}[1]{\textcolor[rgb]{0.25,0.63,0.44}{{#1}}}
    \newcommand{\CharTok}[1]{\textcolor[rgb]{0.25,0.44,0.63}{{#1}}}
    \newcommand{\StringTok}[1]{\textcolor[rgb]{0.25,0.44,0.63}{{#1}}}
    \newcommand{\CommentTok}[1]{\textcolor[rgb]{0.38,0.63,0.69}{\textit{{#1}}}}
    \newcommand{\OtherTok}[1]{\textcolor[rgb]{0.00,0.44,0.13}{{#1}}}
    \newcommand{\AlertTok}[1]{\textcolor[rgb]{1.00,0.00,0.00}{\textbf{{#1}}}}
    \newcommand{\FunctionTok}[1]{\textcolor[rgb]{0.02,0.16,0.49}{{#1}}}
    \newcommand{\RegionMarkerTok}[1]{{#1}}
    \newcommand{\ErrorTok}[1]{\textcolor[rgb]{1.00,0.00,0.00}{\textbf{{#1}}}}
    \newcommand{\NormalTok}[1]{{#1}}
    
    % Additional commands for more recent versions of Pandoc
    \newcommand{\ConstantTok}[1]{\textcolor[rgb]{0.53,0.00,0.00}{{#1}}}
    \newcommand{\SpecialCharTok}[1]{\textcolor[rgb]{0.25,0.44,0.63}{{#1}}}
    \newcommand{\VerbatimStringTok}[1]{\textcolor[rgb]{0.25,0.44,0.63}{{#1}}}
    \newcommand{\SpecialStringTok}[1]{\textcolor[rgb]{0.73,0.40,0.53}{{#1}}}
    \newcommand{\ImportTok}[1]{{#1}}
    \newcommand{\DocumentationTok}[1]{\textcolor[rgb]{0.73,0.13,0.13}{\textit{{#1}}}}
    \newcommand{\AnnotationTok}[1]{\textcolor[rgb]{0.38,0.63,0.69}{\textbf{\textit{{#1}}}}}
    \newcommand{\CommentVarTok}[1]{\textcolor[rgb]{0.38,0.63,0.69}{\textbf{\textit{{#1}}}}}
    \newcommand{\VariableTok}[1]{\textcolor[rgb]{0.10,0.09,0.49}{{#1}}}
    \newcommand{\ControlFlowTok}[1]{\textcolor[rgb]{0.00,0.44,0.13}{\textbf{{#1}}}}
    \newcommand{\OperatorTok}[1]{\textcolor[rgb]{0.40,0.40,0.40}{{#1}}}
    \newcommand{\BuiltInTok}[1]{{#1}}
    \newcommand{\ExtensionTok}[1]{{#1}}
    \newcommand{\PreprocessorTok}[1]{\textcolor[rgb]{0.74,0.48,0.00}{{#1}}}
    \newcommand{\AttributeTok}[1]{\textcolor[rgb]{0.49,0.56,0.16}{{#1}}}
    \newcommand{\InformationTok}[1]{\textcolor[rgb]{0.38,0.63,0.69}{\textbf{\textit{{#1}}}}}
    \newcommand{\WarningTok}[1]{\textcolor[rgb]{0.38,0.63,0.69}{\textbf{\textit{{#1}}}}}
    
    
    % Define a nice break command that doesn't care if a line doesn't already
    % exist.
    \def\br{\hspace*{\fill} \\* }
    % Math Jax compatability definitions
    \def\gt{>}
    \def\lt{<}
    % Document parameters
    \title{Lambda Calculus with the Birds !}
    
    
    

    % Pygments definitions
    
\makeatletter
\def\PY@reset{\let\PY@it=\relax \let\PY@bf=\relax%
    \let\PY@ul=\relax \let\PY@tc=\relax%
    \let\PY@bc=\relax \let\PY@ff=\relax}
\def\PY@tok#1{\csname PY@tok@#1\endcsname}
\def\PY@toks#1+{\ifx\relax#1\empty\else%
    \PY@tok{#1}\expandafter\PY@toks\fi}
\def\PY@do#1{\PY@bc{\PY@tc{\PY@ul{%
    \PY@it{\PY@bf{\PY@ff{#1}}}}}}}
\def\PY#1#2{\PY@reset\PY@toks#1+\relax+\PY@do{#2}}

\expandafter\def\csname PY@tok@w\endcsname{\def\PY@tc##1{\textcolor[rgb]{0.73,0.73,0.73}{##1}}}
\expandafter\def\csname PY@tok@c\endcsname{\let\PY@it=\textit\def\PY@tc##1{\textcolor[rgb]{0.25,0.50,0.50}{##1}}}
\expandafter\def\csname PY@tok@cp\endcsname{\def\PY@tc##1{\textcolor[rgb]{0.74,0.48,0.00}{##1}}}
\expandafter\def\csname PY@tok@k\endcsname{\let\PY@bf=\textbf\def\PY@tc##1{\textcolor[rgb]{0.00,0.50,0.00}{##1}}}
\expandafter\def\csname PY@tok@kp\endcsname{\def\PY@tc##1{\textcolor[rgb]{0.00,0.50,0.00}{##1}}}
\expandafter\def\csname PY@tok@kt\endcsname{\def\PY@tc##1{\textcolor[rgb]{0.69,0.00,0.25}{##1}}}
\expandafter\def\csname PY@tok@o\endcsname{\def\PY@tc##1{\textcolor[rgb]{0.40,0.40,0.40}{##1}}}
\expandafter\def\csname PY@tok@ow\endcsname{\let\PY@bf=\textbf\def\PY@tc##1{\textcolor[rgb]{0.67,0.13,1.00}{##1}}}
\expandafter\def\csname PY@tok@nb\endcsname{\def\PY@tc##1{\textcolor[rgb]{0.00,0.50,0.00}{##1}}}
\expandafter\def\csname PY@tok@nf\endcsname{\def\PY@tc##1{\textcolor[rgb]{0.00,0.00,1.00}{##1}}}
\expandafter\def\csname PY@tok@nc\endcsname{\let\PY@bf=\textbf\def\PY@tc##1{\textcolor[rgb]{0.00,0.00,1.00}{##1}}}
\expandafter\def\csname PY@tok@nn\endcsname{\let\PY@bf=\textbf\def\PY@tc##1{\textcolor[rgb]{0.00,0.00,1.00}{##1}}}
\expandafter\def\csname PY@tok@ne\endcsname{\let\PY@bf=\textbf\def\PY@tc##1{\textcolor[rgb]{0.82,0.25,0.23}{##1}}}
\expandafter\def\csname PY@tok@nv\endcsname{\def\PY@tc##1{\textcolor[rgb]{0.10,0.09,0.49}{##1}}}
\expandafter\def\csname PY@tok@no\endcsname{\def\PY@tc##1{\textcolor[rgb]{0.53,0.00,0.00}{##1}}}
\expandafter\def\csname PY@tok@nl\endcsname{\def\PY@tc##1{\textcolor[rgb]{0.63,0.63,0.00}{##1}}}
\expandafter\def\csname PY@tok@ni\endcsname{\let\PY@bf=\textbf\def\PY@tc##1{\textcolor[rgb]{0.60,0.60,0.60}{##1}}}
\expandafter\def\csname PY@tok@na\endcsname{\def\PY@tc##1{\textcolor[rgb]{0.49,0.56,0.16}{##1}}}
\expandafter\def\csname PY@tok@nt\endcsname{\let\PY@bf=\textbf\def\PY@tc##1{\textcolor[rgb]{0.00,0.50,0.00}{##1}}}
\expandafter\def\csname PY@tok@nd\endcsname{\def\PY@tc##1{\textcolor[rgb]{0.67,0.13,1.00}{##1}}}
\expandafter\def\csname PY@tok@s\endcsname{\def\PY@tc##1{\textcolor[rgb]{0.73,0.13,0.13}{##1}}}
\expandafter\def\csname PY@tok@sd\endcsname{\let\PY@it=\textit\def\PY@tc##1{\textcolor[rgb]{0.73,0.13,0.13}{##1}}}
\expandafter\def\csname PY@tok@si\endcsname{\let\PY@bf=\textbf\def\PY@tc##1{\textcolor[rgb]{0.73,0.40,0.53}{##1}}}
\expandafter\def\csname PY@tok@se\endcsname{\let\PY@bf=\textbf\def\PY@tc##1{\textcolor[rgb]{0.73,0.40,0.13}{##1}}}
\expandafter\def\csname PY@tok@sr\endcsname{\def\PY@tc##1{\textcolor[rgb]{0.73,0.40,0.53}{##1}}}
\expandafter\def\csname PY@tok@ss\endcsname{\def\PY@tc##1{\textcolor[rgb]{0.10,0.09,0.49}{##1}}}
\expandafter\def\csname PY@tok@sx\endcsname{\def\PY@tc##1{\textcolor[rgb]{0.00,0.50,0.00}{##1}}}
\expandafter\def\csname PY@tok@m\endcsname{\def\PY@tc##1{\textcolor[rgb]{0.40,0.40,0.40}{##1}}}
\expandafter\def\csname PY@tok@gh\endcsname{\let\PY@bf=\textbf\def\PY@tc##1{\textcolor[rgb]{0.00,0.00,0.50}{##1}}}
\expandafter\def\csname PY@tok@gu\endcsname{\let\PY@bf=\textbf\def\PY@tc##1{\textcolor[rgb]{0.50,0.00,0.50}{##1}}}
\expandafter\def\csname PY@tok@gd\endcsname{\def\PY@tc##1{\textcolor[rgb]{0.63,0.00,0.00}{##1}}}
\expandafter\def\csname PY@tok@gi\endcsname{\def\PY@tc##1{\textcolor[rgb]{0.00,0.63,0.00}{##1}}}
\expandafter\def\csname PY@tok@gr\endcsname{\def\PY@tc##1{\textcolor[rgb]{1.00,0.00,0.00}{##1}}}
\expandafter\def\csname PY@tok@ge\endcsname{\let\PY@it=\textit}
\expandafter\def\csname PY@tok@gs\endcsname{\let\PY@bf=\textbf}
\expandafter\def\csname PY@tok@gp\endcsname{\let\PY@bf=\textbf\def\PY@tc##1{\textcolor[rgb]{0.00,0.00,0.50}{##1}}}
\expandafter\def\csname PY@tok@go\endcsname{\def\PY@tc##1{\textcolor[rgb]{0.53,0.53,0.53}{##1}}}
\expandafter\def\csname PY@tok@gt\endcsname{\def\PY@tc##1{\textcolor[rgb]{0.00,0.27,0.87}{##1}}}
\expandafter\def\csname PY@tok@err\endcsname{\def\PY@bc##1{\setlength{\fboxsep}{0pt}\fcolorbox[rgb]{1.00,0.00,0.00}{1,1,1}{\strut ##1}}}
\expandafter\def\csname PY@tok@kc\endcsname{\let\PY@bf=\textbf\def\PY@tc##1{\textcolor[rgb]{0.00,0.50,0.00}{##1}}}
\expandafter\def\csname PY@tok@kd\endcsname{\let\PY@bf=\textbf\def\PY@tc##1{\textcolor[rgb]{0.00,0.50,0.00}{##1}}}
\expandafter\def\csname PY@tok@kn\endcsname{\let\PY@bf=\textbf\def\PY@tc##1{\textcolor[rgb]{0.00,0.50,0.00}{##1}}}
\expandafter\def\csname PY@tok@kr\endcsname{\let\PY@bf=\textbf\def\PY@tc##1{\textcolor[rgb]{0.00,0.50,0.00}{##1}}}
\expandafter\def\csname PY@tok@bp\endcsname{\def\PY@tc##1{\textcolor[rgb]{0.00,0.50,0.00}{##1}}}
\expandafter\def\csname PY@tok@fm\endcsname{\def\PY@tc##1{\textcolor[rgb]{0.00,0.00,1.00}{##1}}}
\expandafter\def\csname PY@tok@vc\endcsname{\def\PY@tc##1{\textcolor[rgb]{0.10,0.09,0.49}{##1}}}
\expandafter\def\csname PY@tok@vg\endcsname{\def\PY@tc##1{\textcolor[rgb]{0.10,0.09,0.49}{##1}}}
\expandafter\def\csname PY@tok@vi\endcsname{\def\PY@tc##1{\textcolor[rgb]{0.10,0.09,0.49}{##1}}}
\expandafter\def\csname PY@tok@vm\endcsname{\def\PY@tc##1{\textcolor[rgb]{0.10,0.09,0.49}{##1}}}
\expandafter\def\csname PY@tok@sa\endcsname{\def\PY@tc##1{\textcolor[rgb]{0.73,0.13,0.13}{##1}}}
\expandafter\def\csname PY@tok@sb\endcsname{\def\PY@tc##1{\textcolor[rgb]{0.73,0.13,0.13}{##1}}}
\expandafter\def\csname PY@tok@sc\endcsname{\def\PY@tc##1{\textcolor[rgb]{0.73,0.13,0.13}{##1}}}
\expandafter\def\csname PY@tok@dl\endcsname{\def\PY@tc##1{\textcolor[rgb]{0.73,0.13,0.13}{##1}}}
\expandafter\def\csname PY@tok@s2\endcsname{\def\PY@tc##1{\textcolor[rgb]{0.73,0.13,0.13}{##1}}}
\expandafter\def\csname PY@tok@sh\endcsname{\def\PY@tc##1{\textcolor[rgb]{0.73,0.13,0.13}{##1}}}
\expandafter\def\csname PY@tok@s1\endcsname{\def\PY@tc##1{\textcolor[rgb]{0.73,0.13,0.13}{##1}}}
\expandafter\def\csname PY@tok@mb\endcsname{\def\PY@tc##1{\textcolor[rgb]{0.40,0.40,0.40}{##1}}}
\expandafter\def\csname PY@tok@mf\endcsname{\def\PY@tc##1{\textcolor[rgb]{0.40,0.40,0.40}{##1}}}
\expandafter\def\csname PY@tok@mh\endcsname{\def\PY@tc##1{\textcolor[rgb]{0.40,0.40,0.40}{##1}}}
\expandafter\def\csname PY@tok@mi\endcsname{\def\PY@tc##1{\textcolor[rgb]{0.40,0.40,0.40}{##1}}}
\expandafter\def\csname PY@tok@il\endcsname{\def\PY@tc##1{\textcolor[rgb]{0.40,0.40,0.40}{##1}}}
\expandafter\def\csname PY@tok@mo\endcsname{\def\PY@tc##1{\textcolor[rgb]{0.40,0.40,0.40}{##1}}}
\expandafter\def\csname PY@tok@ch\endcsname{\let\PY@it=\textit\def\PY@tc##1{\textcolor[rgb]{0.25,0.50,0.50}{##1}}}
\expandafter\def\csname PY@tok@cm\endcsname{\let\PY@it=\textit\def\PY@tc##1{\textcolor[rgb]{0.25,0.50,0.50}{##1}}}
\expandafter\def\csname PY@tok@cpf\endcsname{\let\PY@it=\textit\def\PY@tc##1{\textcolor[rgb]{0.25,0.50,0.50}{##1}}}
\expandafter\def\csname PY@tok@c1\endcsname{\let\PY@it=\textit\def\PY@tc##1{\textcolor[rgb]{0.25,0.50,0.50}{##1}}}
\expandafter\def\csname PY@tok@cs\endcsname{\let\PY@it=\textit\def\PY@tc##1{\textcolor[rgb]{0.25,0.50,0.50}{##1}}}

\def\PYZbs{\char`\\}
\def\PYZus{\char`\_}
\def\PYZob{\char`\{}
\def\PYZcb{\char`\}}
\def\PYZca{\char`\^}
\def\PYZam{\char`\&}
\def\PYZlt{\char`\<}
\def\PYZgt{\char`\>}
\def\PYZsh{\char`\#}
\def\PYZpc{\char`\%}
\def\PYZdl{\char`$}
\def\PYZhy{\char`\-}
\def\PYZsq{\char`\'}
\def\PYZdq{\char`\"}
\def\PYZti{\char`\~}
% for compatibility with earlier versions
\def\PYZat{@}
\def\PYZlb{[}
\def\PYZrb{]}
\makeatother


            \makeatletter
\@addtoreset{section}{part}
\def\@part[#1]#2{%
    \ifnum \c@secnumdepth >\m@ne
      \refstepcounter{part}%
      \addcontentsline{toc}{part}{\thepart\hspace{1em}#1}%
    \else
      \addcontentsline{toc}{part}{#1}%
    \fi
    {\parindent \z@ \raggedright
     \interlinepenalty \@M
     \normalfont\centering
     \ifnum \c@secnumdepth >\m@ne
       \LARGE\bfseries \partname\nobreakspace\thepart
       \par\nobreak
     \fi
     \huge \bfseries #2%
     \markboth{}{}\par}%
    \nobreak
    \vskip 3ex
    \@afterheading}
\renewcommand\partname{Topic}
\makeatother

    % Exact colors from NB
    \definecolor{incolor}{rgb}{0.0, 0.0, 0.5}
    \definecolor{outcolor}{rgb}{0.545, 0.0, 0.0}



    
    % Prevent overflowing lines due to hard-to-break entities
    \sloppy 
    % Setup hyperref package
    \hypersetup{
      breaklinks=true,  % so long urls are correctly broken across lines
      colorlinks=true,
      urlcolor=urlcolor,
      linkcolor=linkcolor,
      citecolor=citecolor,
      }
    % Slightly bigger margins than the latex defaults
    
    \geometry{verbose,tmargin=1in,bmargin=1in,lmargin=1in,rmargin=1in}
    
    \author{Satyajit Ghana}

    \begin{document}
    
    
    \maketitle
    
    

    
   % \section{Lambda Calculus with Birds}\label{lambda-calculus-with-birds}
\part{Introduction to the Birds}
    \section{Idiot}\label{idiot}

    The Identity \[ \lambda a.a \]

    \begin{Verbatim}[commandchars=\\\{\}]
{\color{incolor}In [{\color{incolor}1}]:} \PY{n+nx}{I} \PY{o}{=} \PY{n+nx}{a} \PY{p}{=\PYZgt{}} \PY{n+nx}{a}
\end{Verbatim}


\begin{Verbatim}[commandchars=\\\{\}]
{\color{outcolor}Out[{\color{outcolor}1}]:} [Function: I]
\end{Verbatim}
            
    \section{Mocking Bird}\label{mocking-bird}

    The Self Applicator \[ \lambda f. f f \]

    \begin{Verbatim}[commandchars=\\\{\}]
{\color{incolor}In [{\color{incolor}2}]:} \PY{n+nx}{M} \PY{o}{=} \PY{n+nx}{f} \PY{p}{=\PYZgt{}} \PY{n+nx}{f}\PY{p}{(}\PY{n+nx}{f}\PY{p}{)}
\end{Verbatim}


\begin{Verbatim}[commandchars=\\\{\}]
{\color{outcolor}Out[{\color{outcolor}2}]:} [Function: M]
\end{Verbatim}
            
    \section{Kerstrel}\label{kerstrel}

    The Truth \[ \lambda a b . a \]

    \begin{Verbatim}[commandchars=\\\{\}]
{\color{incolor}In [{\color{incolor}3}]:} \PY{n+nx}{K} \PY{o}{=} \PY{n+nx}{a} \PY{p}{=\PYZgt{}} \PY{n+nx}{b} \PY{p}{=\PYZgt{}} \PY{n+nx}{a}
\end{Verbatim}


\begin{Verbatim}[commandchars=\\\{\}]
{\color{outcolor}Out[{\color{outcolor}3}]:} [Function: K]
\end{Verbatim}
            
    \begin{Verbatim}[commandchars=\\\{\}]
{\color{incolor}In [{\color{incolor}4}]:} \PY{n+nx}{K}\PY{p}{.}\PY{n+nx}{inspect} \PY{o}{=} \PY{p}{(}\PY{p}{)} \PY{p}{=\PYZgt{}} \PY{l+s+s1}{\PYZsq{}T / K\PYZsq{}}
\end{Verbatim}


\begin{Verbatim}[commandchars=\\\{\}]
{\color{outcolor}Out[{\color{outcolor}4}]:} [Function]
\end{Verbatim}
            
    \begin{Verbatim}[commandchars=\\\{\}]
{\color{incolor}In [{\color{incolor}5}]:} \PY{n+nx}{K}
\end{Verbatim}


\begin{Verbatim}[commandchars=\\\{\}]
{\color{outcolor}Out[{\color{outcolor}5}]:} T / K
\end{Verbatim}
            
    \section{Kite}\label{kite}

    The False \[ \lambda a b.b = \text{K I} = \text{C K}\]

    \begin{Verbatim}[commandchars=\\\{\}]
{\color{incolor}In [{\color{incolor}6}]:} \PY{n+nx}{KI} \PY{o}{=} \PY{n+nx}{a} \PY{p}{=\PYZgt{}} \PY{n+nx}{b} \PY{p}{=\PYZgt{}} \PY{n+nx}{b}
\end{Verbatim}


\begin{Verbatim}[commandchars=\\\{\}]
{\color{outcolor}Out[{\color{outcolor}6}]:} [Function: KI]
\end{Verbatim}
            
    \begin{Verbatim}[commandchars=\\\{\}]
{\color{incolor}In [{\color{incolor}7}]:} \PY{n+nx}{KI}\PY{p}{.}\PY{n+nx}{inspect} \PY{o}{=} \PY{p}{(}\PY{p}{)} \PY{p}{=\PYZgt{}} \PY{l+s+s1}{\PYZsq{}F / KI\PYZsq{}}
\end{Verbatim}


\begin{Verbatim}[commandchars=\\\{\}]
{\color{outcolor}Out[{\color{outcolor}7}]:} [Function]
\end{Verbatim}
            
    \section{Cardinal}\label{cardinal}

    The Reverse \[ \lambda f a b.  fb a \]

    \begin{Verbatim}[commandchars=\\\{\}]
{\color{incolor}In [{\color{incolor}8}]:} \PY{n+nx}{C} \PY{o}{=} \PY{n+nx}{f} \PY{p}{=\PYZgt{}} \PY{n+nx}{a} \PY{p}{=\PYZgt{}} \PY{n+nx}{b} \PY{p}{=\PYZgt{}} \PY{n+nx}{f}\PY{p}{(}\PY{n+nx}{b}\PY{p}{)}\PY{p}{(}\PY{n+nx}{a}\PY{p}{)}
\end{Verbatim}


\begin{Verbatim}[commandchars=\\\{\}]
{\color{outcolor}Out[{\color{outcolor}8}]:} [Function: C]
\end{Verbatim}
            
    \section{Blue Bird}\label{blue-bird}

    The Composition \[ \lambda f g a. f(g a) \]

    \begin{Verbatim}[commandchars=\\\{\}]
{\color{incolor}In [{\color{incolor}9}]:} \PY{n+nx}{B} \PY{o}{=} \PY{n+nx}{f} \PY{p}{=\PYZgt{}} \PY{n+nx}{g} \PY{p}{=\PYZgt{}} \PY{n+nx}{a} \PY{p}{=\PYZgt{}} \PY{n+nx}{f}\PY{p}{(}\PY{n+nx}{g}\PY{p}{(}\PY{n+nx}{a}\PY{p}{)}\PY{p}{)}
\end{Verbatim}


\begin{Verbatim}[commandchars=\\\{\}]
{\color{outcolor}Out[{\color{outcolor}9}]:} [Function: B]
\end{Verbatim}
            
    \section{Thrush}\label{thrush}

    The Flipper \[ \lambda af. fa  = \text{C I}\]

    \begin{Verbatim}[commandchars=\\\{\}]
{\color{incolor}In [{\color{incolor}10}]:} \PY{n+nx}{T} \PY{o}{=} \PY{n+nx}{a} \PY{p}{=\PYZgt{}} \PY{n+nx}{f} \PY{p}{=\PYZgt{}} \PY{n+nx}{f}\PY{p}{(}\PY{n+nx}{a}\PY{p}{)}
\end{Verbatim}


\begin{Verbatim}[commandchars=\\\{\}]
{\color{outcolor}Out[{\color{outcolor}10}]:} [Function: T]
\end{Verbatim}
            
    \section{Vireo}\label{vireo}

    The Smallest Data Structure \[ \lambda abf. fab = \text{B C T}\]

    \begin{Verbatim}[commandchars=\\\{\}]
{\color{incolor}In [{\color{incolor}11}]:} \PY{n+nx}{V} \PY{o}{=} \PY{n+nx}{a} \PY{p}{=\PYZgt{}} \PY{n+nx}{b} \PY{p}{=\PYZgt{}} \PY{n+nx}{f} \PY{p}{=\PYZgt{}} \PY{n+nx}{f}\PY{p}{(}\PY{n+nx}{a}\PY{p}{)}\PY{p}{(}\PY{n+nx}{b}\PY{p}{)}
\end{Verbatim}


\begin{Verbatim}[commandchars=\\\{\}]
{\color{outcolor}Out[{\color{outcolor}11}]:} [Function: V]
\end{Verbatim}
            
    \section{Black Bird}\label{black-bird}

    Blue Bird for a function that takes two arguments\\
It's the composition of composition of composition
\[ \lambda fgab. f(gab) = \text{B B B}\]

    \begin{Verbatim}[commandchars=\\\{\}]
{\color{incolor}In [{\color{incolor}12}]:} \PY{n+nx}{B1} \PY{o}{=} \PY{n+nx}{f} \PY{p}{=\PYZgt{}} \PY{n+nx}{g} \PY{p}{=\PYZgt{}} \PY{n+nx}{a} \PY{p}{=\PYZgt{}} \PY{n+nx}{b} \PY{p}{=\PYZgt{}} \PY{n+nx}{f}\PY{p}{(}\PY{n+nx}{g}\PY{p}{(}\PY{n+nx}{a}\PY{p}{)}\PY{p}{(}\PY{n+nx}{b}\PY{p}{)}\PY{p}{)}
\end{Verbatim}


\begin{Verbatim}[commandchars=\\\{\}]
{\color{outcolor}Out[{\color{outcolor}12}]:} [Function: B1]
\end{Verbatim}

    \part{Birds ! Birds ! Birds !}
    \section{Fun with these Birds !}\label{fun-with-these-birds}
    %\chapter{Fun with these Birds}

    \begin{Verbatim}[commandchars=\\\{\}]
{\color{incolor}In [{\color{incolor}13}]:} \PY{n+nx}{C}\PY{p}{(}\PY{n+nx}{K}\PY{p}{)}\PY{p}{(}\PY{l+s+s1}{\PYZsq{}T\PYZsq{}}\PY{p}{)}\PY{p}{(}\PY{l+s+s1}{\PYZsq{}F\PYZsq{}}\PY{p}{)}
\end{Verbatim}


\begin{Verbatim}[commandchars=\\\{\}]
{\color{outcolor}Out[{\color{outcolor}13}]:} 'F'
\end{Verbatim}
            
    \begin{Verbatim}[commandchars=\\\{\}]
{\color{incolor}In [{\color{incolor}14}]:} \PY{n+nx}{C}\PY{p}{(}\PY{n+nx}{KI}\PY{p}{)}\PY{p}{(}\PY{l+s+s1}{\PYZsq{}T\PYZsq{}}\PY{p}{)}\PY{p}{(}\PY{l+s+s1}{\PYZsq{}F\PYZsq{}}\PY{p}{)}
\end{Verbatim}


\begin{Verbatim}[commandchars=\\\{\}]
{\color{outcolor}Out[{\color{outcolor}14}]:} 'T'
\end{Verbatim}
            
    Cardinal of Kerstrel is Kite\\
Cardinal of Kite is Kerstrel

    \section{Church Encodings}\label{church-encodings}

    \subsection{Booleans}\label{booleans}

    $ \text{TRUE} = \lambda a b. a = \text{K}$

    \begin{Verbatim}[commandchars=\\\{\}]
{\color{incolor}In [{\color{incolor}15}]:} \PY{n+nx}{TRUE} \PY{o}{=} \PY{n+nx}{K}
\end{Verbatim}


\begin{Verbatim}[commandchars=\\\{\}]
{\color{outcolor}Out[{\color{outcolor}15}]:} T / K
\end{Verbatim}
            
    $ \text{FALSE} = \lambda ab. b = \text{KI} = \text{CK}$

    \begin{Verbatim}[commandchars=\\\{\}]
{\color{incolor}In [{\color{incolor}16}]:} \PY{n+nx}{FALSE} \PY{o}{=} \PY{n+nx}{KI}
\end{Verbatim}


\begin{Verbatim}[commandchars=\\\{\}]
{\color{outcolor}Out[{\color{outcolor}16}]:} F / KI
\end{Verbatim}
            
    $ \text{NOT} = \lambda p. pFT = \text{C}$

    \begin{Verbatim}[commandchars=\\\{\}]
{\color{incolor}In [{\color{incolor}17}]:} \PY{n+nx}{NOT} \PY{o}{=} \PY{n+nx}{C}
\end{Verbatim}


\begin{Verbatim}[commandchars=\\\{\}]
{\color{outcolor}Out[{\color{outcolor}17}]:} [Function: C]
\end{Verbatim}
            
    \begin{Verbatim}[commandchars=\\\{\}]
{\color{incolor}In [{\color{incolor}18}]:} \PY{n+nx}{NOT}\PY{p}{(}\PY{n+nx}{TRUE}\PY{p}{)}\PY{p}{(}\PY{l+s+s1}{\PYZsq{}T\PYZsq{}}\PY{p}{)}\PY{p}{(}\PY{l+s+s1}{\PYZsq{}F\PYZsq{}}\PY{p}{)}
\end{Verbatim}


\begin{Verbatim}[commandchars=\\\{\}]
{\color{outcolor}Out[{\color{outcolor}18}]:} 'F'
\end{Verbatim}
            
    $ \text{AND} = \lambda p q. pqF = \lambda p q . pqp$

    if p is false and it selects false, then p can select itself

    \begin{Verbatim}[commandchars=\\\{\}]
{\color{incolor}In [{\color{incolor}19}]:} \PY{n+nx}{AND} \PY{o}{=} \PY{n+nx}{p} \PY{p}{=\PYZgt{}} \PY{n+nx}{q} \PY{p}{=\PYZgt{}} \PY{n+nx}{p}\PY{p}{(}\PY{n+nx}{q}\PY{p}{)}\PY{p}{(}\PY{n+nx}{p}\PY{p}{)}
\end{Verbatim}


\begin{Verbatim}[commandchars=\\\{\}]
{\color{outcolor}Out[{\color{outcolor}19}]:} [Function: AND]
\end{Verbatim}
            
    \begin{Verbatim}[commandchars=\\\{\}]
{\color{incolor}In [{\color{incolor}20}]:} \PY{n+nx}{AND}\PY{p}{(}\PY{n+nx}{TRUE}\PY{p}{)}\PY{p}{(}\PY{n+nx}{TRUE}\PY{p}{)}
\end{Verbatim}


\begin{Verbatim}[commandchars=\\\{\}]
{\color{outcolor}Out[{\color{outcolor}20}]:} T / K
\end{Verbatim}
            
    \begin{Verbatim}[commandchars=\\\{\}]
{\color{incolor}In [{\color{incolor}21}]:} \PY{n+nx}{AND}\PY{p}{(}\PY{n+nx}{TRUE}\PY{p}{)}\PY{p}{(}\PY{n+nx}{FALSE}\PY{p}{)}
\end{Verbatim}


\begin{Verbatim}[commandchars=\\\{\}]
{\color{outcolor}Out[{\color{outcolor}21}]:} F / KI
\end{Verbatim}
            
    \begin{Verbatim}[commandchars=\\\{\}]
{\color{incolor}In [{\color{incolor}22}]:} \PY{n+nx}{AND}\PY{p}{(}\PY{n+nx}{FALSE}\PY{p}{)}\PY{p}{(}\PY{n+nx}{FALSE}\PY{p}{)}
\end{Verbatim}


\begin{Verbatim}[commandchars=\\\{\}]
{\color{outcolor}Out[{\color{outcolor}22}]:} F / KI
\end{Verbatim}
            
    $ \text{OR} = \lambda p q. pTq = \lambda p q. p p q = \lambda p q. M q
= \text{M}$

    \begin{Verbatim}[commandchars=\\\{\}]
{\color{incolor}In [{\color{incolor}23}]:} \PY{n+nx}{OR} \PY{o}{=} \PY{n+nx}{p} \PY{p}{=\PYZgt{}} \PY{n+nx}{q} \PY{p}{=\PYZgt{}} \PY{n+nx}{M}\PY{p}{(}\PY{n+nx}{p}\PY{p}{)}\PY{p}{(}\PY{n+nx}{q}\PY{p}{)}
\end{Verbatim}


\begin{Verbatim}[commandchars=\\\{\}]
{\color{outcolor}Out[{\color{outcolor}23}]:} [Function: OR]
\end{Verbatim}
            
    \begin{Verbatim}[commandchars=\\\{\}]
{\color{incolor}In [{\color{incolor}24}]:} \PY{n+nx}{OR}\PY{p}{(}\PY{n+nx}{TRUE}\PY{p}{)}\PY{p}{(}\PY{n+nx}{TRUE}\PY{p}{)}
\end{Verbatim}


\begin{Verbatim}[commandchars=\\\{\}]
{\color{outcolor}Out[{\color{outcolor}24}]:} T / K
\end{Verbatim}
            
    \begin{Verbatim}[commandchars=\\\{\}]
{\color{incolor}In [{\color{incolor}25}]:} \PY{n+nx}{OR}\PY{p}{(}\PY{n+nx}{TRUE}\PY{p}{)}\PY{p}{(}\PY{n+nx}{FALSE}\PY{p}{)}
\end{Verbatim}


\begin{Verbatim}[commandchars=\\\{\}]
{\color{outcolor}Out[{\color{outcolor}25}]:} T / K
\end{Verbatim}
            
    \begin{Verbatim}[commandchars=\\\{\}]
{\color{incolor}In [{\color{incolor}26}]:} \PY{n+nx}{OR}\PY{p}{(}\PY{n+nx}{FALSE}\PY{p}{)}\PY{p}{(}\PY{n+nx}{FALSE}\PY{p}{)}
\end{Verbatim}


\begin{Verbatim}[commandchars=\\\{\}]
{\color{outcolor}Out[{\color{outcolor}26}]:} F / KI
\end{Verbatim}
            
    \((\lambda pq.ppq)xy = xxy\)\\
\(M x y = xxy\)\\
\(\text{OR} = \text{M}\)

    \begin{Verbatim}[commandchars=\\\{\}]
{\color{incolor}In [{\color{incolor}27}]:} \PY{n+nx}{M}\PY{p}{(}\PY{n+nx}{TRUE}\PY{p}{)}\PY{p}{(}\PY{n+nx}{TRUE}\PY{p}{)}
\end{Verbatim}


\begin{Verbatim}[commandchars=\\\{\}]
{\color{outcolor}Out[{\color{outcolor}27}]:} T / K
\end{Verbatim}
            
    \begin{Verbatim}[commandchars=\\\{\}]
{\color{incolor}In [{\color{incolor}28}]:} \PY{n+nx}{M}\PY{p}{(}\PY{n+nx}{TRUE}\PY{p}{)}\PY{p}{(}\PY{n+nx}{FALSE}\PY{p}{)}
\end{Verbatim}


\begin{Verbatim}[commandchars=\\\{\}]
{\color{outcolor}Out[{\color{outcolor}28}]:} T / K
\end{Verbatim}
            
    \begin{Verbatim}[commandchars=\\\{\}]
{\color{incolor}In [{\color{incolor}29}]:} \PY{n+nx}{M}\PY{p}{(}\PY{n+nx}{FALSE}\PY{p}{)}\PY{p}{(}\PY{n+nx}{FALSE}\PY{p}{)}
\end{Verbatim}


\begin{Verbatim}[commandchars=\\\{\}]
{\color{outcolor}Out[{\color{outcolor}29}]:} F / KI
\end{Verbatim}
            
    \(\text{BEQ} = \lambda pq.pq(\text{NOT}q)\)

    This is also the \texttt{XNOR} or the Equality

    \begin{Verbatim}[commandchars=\\\{\}]
{\color{incolor}In [{\color{incolor}30}]:} \PY{n+nx}{BEQ} \PY{o}{=} \PY{n+nx}{p} \PY{p}{=\PYZgt{}} \PY{n+nx}{q} \PY{p}{=\PYZgt{}} \PY{n+nx}{p}\PY{p}{(}\PY{n+nx}{q}\PY{p}{)}\PY{p}{(}\PY{n+nx}{NOT}\PY{p}{(}\PY{n+nx}{q}\PY{p}{)}\PY{p}{)}
\end{Verbatim}


\begin{Verbatim}[commandchars=\\\{\}]
{\color{outcolor}Out[{\color{outcolor}30}]:} [Function: BEQ]
\end{Verbatim}
            
    \begin{Verbatim}[commandchars=\\\{\}]
{\color{incolor}In [{\color{incolor}31}]:} \PY{n+nx}{BEQ}\PY{p}{(}\PY{n+nx}{TRUE}\PY{p}{)}\PY{p}{(}\PY{n+nx}{TRUE}\PY{p}{)}\PY{p}{(}\PY{l+s+s1}{\PYZsq{}T\PYZsq{}}\PY{p}{)}\PY{p}{(}\PY{l+s+s1}{\PYZsq{}F\PYZsq{}}\PY{p}{)}
\end{Verbatim}


\begin{Verbatim}[commandchars=\\\{\}]
{\color{outcolor}Out[{\color{outcolor}31}]:} 'T'
\end{Verbatim}
            
    \begin{Verbatim}[commandchars=\\\{\}]
{\color{incolor}In [{\color{incolor}32}]:} \PY{n+nx}{BEQ}\PY{p}{(}\PY{n+nx}{FALSE}\PY{p}{)}\PY{p}{(}\PY{n+nx}{FALSE}\PY{p}{)}\PY{p}{(}\PY{l+s+s1}{\PYZsq{}T\PYZsq{}}\PY{p}{)}\PY{p}{(}\PY{l+s+s1}{\PYZsq{}F\PYZsq{}}\PY{p}{)}
\end{Verbatim}


\begin{Verbatim}[commandchars=\\\{\}]
{\color{outcolor}Out[{\color{outcolor}32}]:} 'T'
\end{Verbatim}
            
    \begin{Verbatim}[commandchars=\\\{\}]
{\color{incolor}In [{\color{incolor}33}]:} \PY{n+nx}{BEQ}\PY{p}{(}\PY{n+nx}{FALSE}\PY{p}{)}\PY{p}{(}\PY{n+nx}{TRUE}\PY{p}{)}\PY{p}{(}\PY{l+s+s1}{\PYZsq{}T\PYZsq{}}\PY{p}{)}\PY{p}{(}\PY{l+s+s1}{\PYZsq{}F\PYZsq{}}\PY{p}{)}
\end{Verbatim}


\begin{Verbatim}[commandchars=\\\{\}]
{\color{outcolor}Out[{\color{outcolor}33}]:} 'F'
\end{Verbatim}
            
    \subsection{Numerals}\label{numerals}

    \(\text{ZERO} = \lambda fa.a\)

    \begin{Verbatim}[commandchars=\\\{\}]
{\color{incolor}In [{\color{incolor}34}]:} \PY{n+nx}{ZERO} \PY{o}{=} \PY{n+nx}{f} \PY{p}{=\PYZgt{}} \PY{n+nx}{a}  \PY{p}{=\PYZgt{}} \PY{n+nx}{a}
\end{Verbatim}


\begin{Verbatim}[commandchars=\\\{\}]
{\color{outcolor}Out[{\color{outcolor}34}]:} [Function: ZERO]
\end{Verbatim}
            
    \(\text{ONCE} = \lambda f a . f a\)

    \begin{Verbatim}[commandchars=\\\{\}]
{\color{incolor}In [{\color{incolor}35}]:} \PY{n+nx}{ONCE} \PY{o}{=} \PY{n+nx}{f} \PY{p}{=\PYZgt{}} \PY{n+nx}{a} \PY{p}{=\PYZgt{}} \PY{n+nx}{f}\PY{p}{(}\PY{n+nx}{a}\PY{p}{)}
\end{Verbatim}


\begin{Verbatim}[commandchars=\\\{\}]
{\color{outcolor}Out[{\color{outcolor}35}]:} [Function: ONCE]
\end{Verbatim}
            
    $ \text{TWICE} = \lambda f a . f(f a) $

    \begin{Verbatim}[commandchars=\\\{\}]
{\color{incolor}In [{\color{incolor}36}]:} \PY{n+nx}{TWICE} \PY{o}{=} \PY{n+nx}{f} \PY{p}{=\PYZgt{}} \PY{n+nx}{a} \PY{p}{=\PYZgt{}} \PY{n+nx}{f}\PY{p}{(}\PY{n+nx}{f}\PY{p}{(}\PY{n+nx}{a}\PY{p}{)}\PY{p}{)}
\end{Verbatim}


\begin{Verbatim}[commandchars=\\\{\}]
{\color{outcolor}Out[{\color{outcolor}36}]:} [Function: TWICE]
\end{Verbatim}
            
    $ \text{THRICE} = \lambda fa.f(f(f a) $

    \begin{Verbatim}[commandchars=\\\{\}]
{\color{incolor}In [{\color{incolor}37}]:} \PY{n+nx}{THRICE} \PY{o}{=} \PY{n+nx}{f} \PY{p}{=\PYZgt{}} \PY{n+nx}{a} \PY{p}{=\PYZgt{}} \PY{n+nx}{f}\PY{p}{(}\PY{n+nx}{f}\PY{p}{(}\PY{n+nx}{f}\PY{p}{(}\PY{n+nx}{a}\PY{p}{)}\PY{p}{)}\PY{p}{)}
\end{Verbatim}


\begin{Verbatim}[commandchars=\\\{\}]
{\color{outcolor}Out[{\color{outcolor}37}]:} [Function: THRICE]
\end{Verbatim}
            
    $ \text{FOURFOLD} = \lambda f a. f(f(f(f a))) $

    \begin{Verbatim}[commandchars=\\\{\}]
{\color{incolor}In [{\color{incolor}38}]:} \PY{n+nx}{FOURFOLD} \PY{o}{=} \PY{n+nx}{f} \PY{p}{=\PYZgt{}} \PY{n+nx}{a} \PY{p}{=\PYZgt{}} \PY{n+nx}{f}\PY{p}{(}\PY{n+nx}{f}\PY{p}{(}\PY{n+nx}{f}\PY{p}{(}\PY{n+nx}{f}\PY{p}{(}\PY{n+nx}{a}\PY{p}{)}\PY{p}{)}\PY{p}{)}\PY{p}{)}
\end{Verbatim}


\begin{Verbatim}[commandchars=\\\{\}]
{\color{outcolor}Out[{\color{outcolor}38}]:} [Function: FOURFOLD]
\end{Verbatim}
            
    \begin{Verbatim}[commandchars=\\\{\}]
{\color{incolor}In [{\color{incolor}39}]:} \PY{n+nx}{N0} \PY{o}{=} \PY{n+nx}{ZERO}
         \PY{n+nx}{N1} \PY{o}{=} \PY{n+nx}{ONCE}
         \PY{n+nx}{N2} \PY{o}{=} \PY{n+nx}{TWICE}
         \PY{n+nx}{N3} \PY{o}{=} \PY{n+nx}{THRICE}
         \PY{n+nx}{N4} \PY{o}{=} \PY{n+nx}{FOURFOLD}
\end{Verbatim}


\begin{Verbatim}[commandchars=\\\{\}]
{\color{outcolor}Out[{\color{outcolor}39}]:} [Function: FOURFOLD]
\end{Verbatim}
            
    $ \text{SUCC} = \lambda n f a. f(nfa) = \lambda n f.\text{B}f(nf) $

    \begin{Verbatim}[commandchars=\\\{\}]
{\color{incolor}In [{\color{incolor}40}]:} \PY{n+nx}{SUCC} \PY{o}{=} \PY{n+nx}{n} \PY{p}{=\PYZgt{}} \PY{n+nx}{f} \PY{p}{=\PYZgt{}} \PY{n+nx}{x} \PY{p}{=\PYZgt{}} \PY{n+nx}{f}\PY{p}{(}\PY{n+nx}{n}\PY{p}{(}\PY{n+nx}{f}\PY{p}{)}\PY{p}{(}\PY{n+nx}{x}\PY{p}{)}\PY{p}{)}
\end{Verbatim}


\begin{Verbatim}[commandchars=\\\{\}]
{\color{outcolor}Out[{\color{outcolor}40}]:} [Function: SUCC]
\end{Verbatim}
            
    This is the same as function composition, why not use the Blue Bird ?

    \begin{Verbatim}[commandchars=\\\{\}]
{\color{incolor}In [{\color{incolor}41}]:} \PY{n+nx}{SUCC} \PY{o}{=} \PY{n+nx}{n} \PY{p}{=\PYZgt{}} \PY{n+nx}{f} \PY{p}{=\PYZgt{}} \PY{n+nx}{B}\PY{p}{(}\PY{n+nx}{f}\PY{p}{)}\PY{p}{(}\PY{n+nx}{n}\PY{p}{(}\PY{n+nx}{f}\PY{p}{)}\PY{p}{)}
\end{Verbatim}


\begin{Verbatim}[commandchars=\\\{\}]
{\color{outcolor}Out[{\color{outcolor}41}]:} [Function: SUCC]
\end{Verbatim}
            
    \begin{Verbatim}[commandchars=\\\{\}]
{\color{incolor}In [{\color{incolor}42}]:} \PY{n+nx}{SUCC}\PY{p}{(}\PY{n+nx}{THRICE}\PY{p}{)}\PY{p}{(}\PY{n+nx}{x} \PY{p}{=\PYZgt{}} \PY{n+nx}{x} \PY{o}{+} \PY{l+m+mi}{1}\PY{p}{)}\PY{p}{(}\PY{l+m+mi}{0}\PY{p}{)}
\end{Verbatim}


\begin{Verbatim}[commandchars=\\\{\}]
{\color{outcolor}Out[{\color{outcolor}42}]:} 4
\end{Verbatim}
            
    $ \text{ADD} = \lambda n k. n \text{ SUCC } k $

    \begin{Verbatim}[commandchars=\\\{\}]
{\color{incolor}In [{\color{incolor}43}]:} \PY{n+nx}{ADD} \PY{o}{=} \PY{n+nx}{n} \PY{p}{=\PYZgt{}} \PY{n+nx}{k} \PY{p}{=\PYZgt{}} \PY{n+nx}{n}\PY{p}{(}\PY{n+nx}{SUCC}\PY{p}{)}\PY{p}{(}\PY{n+nx}{k}\PY{p}{)}
\end{Verbatim}


\begin{Verbatim}[commandchars=\\\{\}]
{\color{outcolor}Out[{\color{outcolor}43}]:} [Function: ADD]
\end{Verbatim}
            
    Read it as n times SUCC, applied to k, same as
\texttt{SUCC(SUCC(SUCC(k)))} where SUCC is n times this example was 3
added to k

    \begin{Verbatim}[commandchars=\\\{\}]
{\color{incolor}In [{\color{incolor}44}]:} \PY{n+nx}{ADD}\PY{p}{(}\PY{n+nx}{N3}\PY{p}{)}\PY{p}{(}\PY{n+nx}{N2}\PY{p}{)}\PY{p}{(}\PY{n+nx}{x} \PY{p}{=\PYZgt{}} \PY{n+nx}{x} \PY{o}{+} \PY{l+m+mi}{1}\PY{p}{)}\PY{p}{(}\PY{l+m+mi}{0}\PY{p}{)}
\end{Verbatim}


\begin{Verbatim}[commandchars=\\\{\}]
{\color{outcolor}Out[{\color{outcolor}44}]:} 5
\end{Verbatim}
            
    $ \text{MULT} = \lambda n k f. n(kf) = \lambda n k. \text{B}nk =
\text{B}$

    \begin{Verbatim}[commandchars=\\\{\}]
{\color{incolor}In [{\color{incolor}45}]:} \PY{n+nx}{MULT} \PY{o}{=} \PY{n+nx}{n} \PY{p}{=\PYZgt{}} \PY{n+nx}{k} \PY{p}{=\PYZgt{}} \PY{n+nx}{B}\PY{p}{(}\PY{n+nx}{n}\PY{p}{)}\PY{p}{(}\PY{n+nx}{k}\PY{p}{)}
         \PY{n+nx}{MULT} \PY{o}{=} \PY{n+nx}{B}
\end{Verbatim}


\begin{Verbatim}[commandchars=\\\{\}]
{\color{outcolor}Out[{\color{outcolor}45}]:} [Function: B]
\end{Verbatim}
            
    \begin{Verbatim}[commandchars=\\\{\}]
{\color{incolor}In [{\color{incolor}46}]:} \PY{n+nx}{MULT}\PY{p}{(}\PY{n+nx}{N3}\PY{p}{)}\PY{p}{(}\PY{n+nx}{N2}\PY{p}{)}\PY{p}{(}\PY{n+nx}{x} \PY{p}{=\PYZgt{}} \PY{n+nx}{x} \PY{o}{+} \PY{l+m+mi}{1}\PY{p}{)}\PY{p}{(}\PY{l+m+mi}{0}\PY{p}{)}
\end{Verbatim}


\begin{Verbatim}[commandchars=\\\{\}]
{\color{outcolor}Out[{\color{outcolor}46}]:} 6
\end{Verbatim}
            
    $ \text{POW} = \lambda n k. k \text{ MULT }n = \text{C I} = \text{T}$

    \begin{Verbatim}[commandchars=\\\{\}]
{\color{incolor}In [{\color{incolor}47}]:} \PY{n+nx}{POW} \PY{o}{=} \PY{n+nx}{T}
\end{Verbatim}


\begin{Verbatim}[commandchars=\\\{\}]
{\color{outcolor}Out[{\color{outcolor}47}]:} [Function: T]
\end{Verbatim}
            
    \begin{Verbatim}[commandchars=\\\{\}]
{\color{incolor}In [{\color{incolor}48}]:} \PY{n+nx}{POW}\PY{p}{(}\PY{n+nx}{N3}\PY{p}{)}\PY{p}{(}\PY{n+nx}{N2}\PY{p}{)}\PY{p}{(}\PY{n+nx}{x} \PY{p}{=\PYZgt{}} \PY{n+nx}{x} \PY{o}{+} \PY{l+m+mi}{1}\PY{p}{)}\PY{p}{(}\PY{l+m+mi}{0}\PY{p}{)}
\end{Verbatim}


\begin{Verbatim}[commandchars=\\\{\}]
{\color{outcolor}Out[{\color{outcolor}48}]:} 9
\end{Verbatim}
            
    $ \text{IS0} = \lambda n. n (K F) T $

    KF always gives a False, Kerstrel of False

    \begin{Verbatim}[commandchars=\\\{\}]
{\color{incolor}In [{\color{incolor}49}]:} \PY{n+nx}{IS0} \PY{o}{=} \PY{n+nx}{n} \PY{p}{=\PYZgt{}} \PY{n+nx}{n}\PY{p}{(}\PY{n+nx}{K}\PY{p}{(}\PY{n+nx}{FALSE}\PY{p}{)}\PY{p}{)}\PY{p}{(}\PY{n+nx}{TRUE}\PY{p}{)}
\end{Verbatim}


\begin{Verbatim}[commandchars=\\\{\}]
{\color{outcolor}Out[{\color{outcolor}49}]:} [Function: IS0]
\end{Verbatim}
            
    \begin{Verbatim}[commandchars=\\\{\}]
{\color{incolor}In [{\color{incolor}50}]:} \PY{n+nx}{IS0}\PY{p}{(}\PY{n+nx}{N0}\PY{p}{)}
\end{Verbatim}


\begin{Verbatim}[commandchars=\\\{\}]
{\color{outcolor}Out[{\color{outcolor}50}]:} T / K
\end{Verbatim}
            
    \begin{Verbatim}[commandchars=\\\{\}]
{\color{incolor}In [{\color{incolor}51}]:} \PY{n+nx}{IS0}\PY{p}{(}\PY{n+nx}{N1}\PY{p}{)}
\end{Verbatim}


\begin{Verbatim}[commandchars=\\\{\}]
{\color{outcolor}Out[{\color{outcolor}51}]:} F / KI
\end{Verbatim}
            
    $ \text{PRED} = \lambda n.
n(\lambda g.\text{IS0}(g\text{N1})\text{I}(\text{B SUCC }
g))(\text{K N0})\text{N0}$

    $ \text{V I M} $

    Vireo is the smallest data structure, you box in the two arguments like
f(a)(b) and then whenever you want a value back you send in a function
to recieve eithe a or b

    \begin{Verbatim}[commandchars=\\\{\}]
{\color{incolor}In [{\color{incolor}52}]:} \PY{n+nx}{vim} \PY{o}{=} \PY{n+nx}{V}\PY{p}{(}\PY{n+nx}{I}\PY{p}{)}\PY{p}{(}\PY{n+nx}{M}\PY{p}{)}
\end{Verbatim}


\begin{Verbatim}[commandchars=\\\{\}]
{\color{outcolor}Out[{\color{outcolor}52}]:} [Function]
\end{Verbatim}
            
    \begin{Verbatim}[commandchars=\\\{\}]
{\color{incolor}In [{\color{incolor}53}]:} \PY{n+nx}{vim}\PY{p}{(}\PY{n+nx}{K}\PY{p}{)}
\end{Verbatim}


\begin{Verbatim}[commandchars=\\\{\}]
{\color{outcolor}Out[{\color{outcolor}53}]:} [Function: I]
\end{Verbatim}
            
    \begin{Verbatim}[commandchars=\\\{\}]
{\color{incolor}In [{\color{incolor}54}]:} \PY{n+nx}{vim}\PY{p}{(}\PY{n+nx}{C}\PY{p}{(}\PY{n+nx}{K}\PY{p}{)}\PY{p}{)} \PY{c+c1}{// C(K) = KI}
\end{Verbatim}


\begin{Verbatim}[commandchars=\\\{\}]
{\color{outcolor}Out[{\color{outcolor}54}]:} [Function: M]
\end{Verbatim}
            
    $ \text{PAIR} = \text{V} $

    \begin{Verbatim}[commandchars=\\\{\}]
{\color{incolor}In [{\color{incolor}55}]:} \PY{n+nx}{PAIR} \PY{o}{=} \PY{n+nx}{V}
\end{Verbatim}


\begin{Verbatim}[commandchars=\\\{\}]
{\color{outcolor}Out[{\color{outcolor}55}]:} [Function: V]
\end{Verbatim}
            
    $ \text{FST} = \lambda p. p K $

    \begin{Verbatim}[commandchars=\\\{\}]
{\color{incolor}In [{\color{incolor}56}]:} \PY{n+nx}{FST} \PY{o}{=} \PY{n+nx}{p} \PY{p}{=\PYZgt{}} \PY{n+nx}{p}\PY{p}{(}\PY{n+nx}{K}\PY{p}{)}
\end{Verbatim}


\begin{Verbatim}[commandchars=\\\{\}]
{\color{outcolor}Out[{\color{outcolor}56}]:} [Function: FST]
\end{Verbatim}
            
    \begin{Verbatim}[commandchars=\\\{\}]
{\color{incolor}In [{\color{incolor}57}]:} \PY{n+nx}{FST}\PY{p}{(}\PY{n+nx}{vim}\PY{p}{)}
\end{Verbatim}


\begin{Verbatim}[commandchars=\\\{\}]
{\color{outcolor}Out[{\color{outcolor}57}]:} [Function: I]
\end{Verbatim}
            
    $ \text{SND} = \lambda p. p(KI) $

    \begin{Verbatim}[commandchars=\\\{\}]
{\color{incolor}In [{\color{incolor}58}]:} \PY{n+nx}{SND} \PY{o}{=} \PY{n+nx}{p} \PY{p}{=\PYZgt{}} \PY{n+nx}{p}\PY{p}{(}\PY{n+nx}{KI}\PY{p}{)}
\end{Verbatim}


\begin{Verbatim}[commandchars=\\\{\}]
{\color{outcolor}Out[{\color{outcolor}58}]:} [Function: SND]
\end{Verbatim}
            
    \begin{Verbatim}[commandchars=\\\{\}]
{\color{incolor}In [{\color{incolor}59}]:} \PY{n+nx}{SND}\PY{p}{(}\PY{n+nx}{vim}\PY{p}{)}
\end{Verbatim}


\begin{Verbatim}[commandchars=\\\{\}]
{\color{outcolor}Out[{\color{outcolor}59}]:} [Function: M]
\end{Verbatim}
            
    $ \text{PHI} =
\lambda p.\text{V}(\text{SND }p)(\text{SUCC }(\text{SND})p) $

    copy 2nd to 1st, and increment 2nd

    \begin{Verbatim}[commandchars=\\\{\}]
{\color{incolor}In [{\color{incolor}60}]:} \PY{n+nx}{PHI} \PY{o}{=} \PY{n+nx}{p} \PY{p}{=\PYZgt{}} \PY{n+nx}{V}\PY{p}{(}\PY{n+nx}{SND}\PY{p}{(}\PY{n+nx}{p}\PY{p}{)}\PY{p}{)}\PY{p}{(}\PY{n+nx}{SUCC}\PY{p}{(}\PY{n+nx}{SND}\PY{p}{(}\PY{n+nx}{p}\PY{p}{)}\PY{p}{)}\PY{p}{)}
\end{Verbatim}


\begin{Verbatim}[commandchars=\\\{\}]
{\color{outcolor}Out[{\color{outcolor}60}]:} [Function: PHI]
\end{Verbatim}
            
    \begin{Verbatim}[commandchars=\\\{\}]
{\color{incolor}In [{\color{incolor}61}]:} \PY{n+nx}{SND}\PY{p}{(}\PY{n+nx}{PHI}\PY{p}{(}\PY{n+nx}{V}\PY{p}{(}\PY{n+nx}{M}\PY{p}{)}\PY{p}{(}\PY{n+nx}{N3}\PY{p}{)}\PY{p}{)}\PY{p}{)}\PY{p}{(}\PY{n+nx}{x} \PY{p}{=\PYZgt{}} \PY{n+nx}{x} \PY{o}{+} \PY{l+m+mi}{1}\PY{p}{)}\PY{p}{(}\PY{l+m+mi}{0}\PY{p}{)}
\end{Verbatim}


\begin{Verbatim}[commandchars=\\\{\}]
{\color{outcolor}Out[{\color{outcolor}61}]:} 4
\end{Verbatim}
            
    \begin{Verbatim}[commandchars=\\\{\}]
{\color{incolor}In [{\color{incolor}62}]:} \PY{n+nx}{FST}\PY{p}{(}\PY{n+nx}{PHI}\PY{p}{(}\PY{n+nx}{V}\PY{p}{(}\PY{n+nx}{M}\PY{p}{)}\PY{p}{(}\PY{n+nx}{N3}\PY{p}{)}\PY{p}{)}\PY{p}{)}\PY{p}{(}\PY{n+nx}{x} \PY{p}{=\PYZgt{}} \PY{n+nx}{x} \PY{o}{+} \PY{l+m+mi}{1}\PY{p}{)}\PY{p}{(}\PY{l+m+mi}{0}\PY{p}{)}
\end{Verbatim}


\begin{Verbatim}[commandchars=\\\{\}]
{\color{outcolor}Out[{\color{outcolor}62}]:} 3
\end{Verbatim}
            
    \texttt{N0\ PHI(N0,\ N0)\ =\ (N0,\ N0)}\\
\texttt{N1\ PHI(N0,\ N0)\ =\ (N0,\ N1)}\\
\texttt{N2\ PHI(N0,\ N0)\ =\ (N1,\ N2)}\\
\texttt{...}\\
\texttt{N8\ PHI(N0,\ N0)\ =\ (N7,\ N8)}

    Holy Cow !, you got the predecessor working !

    $ \text{PRED} = \lambda n = \text{FST }(n \Phi (\text{PAIR }
\text{ZERO } \text{ZERO}))$

    \begin{Verbatim}[commandchars=\\\{\}]
{\color{incolor}In [{\color{incolor}63}]:} \PY{n+nx}{PRED} \PY{o}{=} \PY{n+nx}{n} \PY{p}{=\PYZgt{}} \PY{n+nx}{FST}\PY{p}{(}\PY{n+nx}{n}\PY{p}{(}\PY{n+nx}{PHI}\PY{p}{)}\PY{p}{(}\PY{n+nx}{V}\PY{p}{(}\PY{n+nx}{N0}\PY{p}{)}\PY{p}{(}\PY{n+nx}{N0}\PY{p}{)}\PY{p}{)}\PY{p}{)}
\end{Verbatim}


\begin{Verbatim}[commandchars=\\\{\}]
{\color{outcolor}Out[{\color{outcolor}63}]:} [Function: PRED]
\end{Verbatim}
            
    FIRST of "n" application of PHI to PAIR of ZERO, ZERO

    \begin{Verbatim}[commandchars=\\\{\}]
{\color{incolor}In [{\color{incolor}64}]:} \PY{n+nx}{PRED}\PY{p}{(}\PY{n+nx}{N3}\PY{p}{)}\PY{p}{(}\PY{n+nx}{x} \PY{p}{=\PYZgt{}} \PY{n+nx}{x} \PY{o}{+} \PY{l+m+mi}{1}\PY{p}{)}\PY{p}{(}\PY{l+m+mi}{0}\PY{p}{)}
\end{Verbatim}


\begin{Verbatim}[commandchars=\\\{\}]
{\color{outcolor}Out[{\color{outcolor}64}]:} 2
\end{Verbatim}
            
    $ \text{SUB} = \lambda nk. k \text{ PRED } n $

    \begin{Verbatim}[commandchars=\\\{\}]
{\color{incolor}In [{\color{incolor}65}]:} \PY{n+nx}{SUB} \PY{o}{=} \PY{n+nx}{n} \PY{p}{=\PYZgt{}} \PY{n+nx}{k} \PY{p}{=\PYZgt{}} \PY{n+nx}{k}\PY{p}{(}\PY{n+nx}{PRED}\PY{p}{)}\PY{p}{(}\PY{n+nx}{n}\PY{p}{)}
\end{Verbatim}


\begin{Verbatim}[commandchars=\\\{\}]
{\color{outcolor}Out[{\color{outcolor}65}]:} [Function: SUB]
\end{Verbatim}
            
    \begin{Verbatim}[commandchars=\\\{\}]
{\color{incolor}In [{\color{incolor}66}]:} \PY{n+nx}{SUB}\PY{p}{(}\PY{n+nx}{N4}\PY{p}{)}\PY{p}{(}\PY{n+nx}{N3}\PY{p}{)}\PY{p}{(}\PY{n+nx}{x} \PY{p}{=\PYZgt{}} \PY{n+nx}{x} \PY{o}{+} \PY{l+m+mi}{1}\PY{p}{)}\PY{p}{(}\PY{l+m+mi}{0}\PY{p}{)}
\end{Verbatim}


\begin{Verbatim}[commandchars=\\\{\}]
{\color{outcolor}Out[{\color{outcolor}66}]:} 1
\end{Verbatim}
            
    $ \text{LEQ} = \lambda nk.\text{IS0}(\text{SUB }nk) $

    \begin{Verbatim}[commandchars=\\\{\}]
{\color{incolor}In [{\color{incolor}67}]:} \PY{n+nx}{LEQ} \PY{o}{=} \PY{n+nx}{n} \PY{p}{=\PYZgt{}} \PY{n+nx}{k} \PY{p}{=\PYZgt{}} \PY{n+nx}{IS0}\PY{p}{(}\PY{n+nx}{SUB}\PY{p}{(}\PY{n+nx}{n}\PY{p}{)}\PY{p}{(}\PY{n+nx}{k}\PY{p}{)}\PY{p}{)}
\end{Verbatim}


\begin{Verbatim}[commandchars=\\\{\}]
{\color{outcolor}Out[{\color{outcolor}67}]:} [Function: LEQ]
\end{Verbatim}
            
    \begin{Verbatim}[commandchars=\\\{\}]
{\color{incolor}In [{\color{incolor}68}]:} \PY{n+nx}{LEQ}\PY{p}{(}\PY{n+nx}{N3}\PY{p}{)}\PY{p}{(}\PY{n+nx}{N4}\PY{p}{)}
\end{Verbatim}


\begin{Verbatim}[commandchars=\\\{\}]
{\color{outcolor}Out[{\color{outcolor}68}]:} T / K
\end{Verbatim}
            
    \begin{Verbatim}[commandchars=\\\{\}]
{\color{incolor}In [{\color{incolor}69}]:} \PY{n+nx}{LEQ}\PY{p}{(}\PY{n+nx}{N4}\PY{p}{)}\PY{p}{(}\PY{n+nx}{N3}\PY{p}{)}
\end{Verbatim}


\begin{Verbatim}[commandchars=\\\{\}]
{\color{outcolor}Out[{\color{outcolor}69}]:} F / KI
\end{Verbatim}
            
    $ \text{EQ} = \lambda nk. \text{AND}(\text{LEQ}nk)(\text{LEQ}kn)$

    \begin{Verbatim}[commandchars=\\\{\}]
{\color{incolor}In [{\color{incolor}70}]:} \PY{n+nx}{EQ} \PY{o}{=} \PY{n+nx}{n} \PY{p}{=\PYZgt{}} \PY{n+nx}{k} \PY{p}{=\PYZgt{}} \PY{n+nx}{AND}\PY{p}{(}\PY{n+nx}{LEQ}\PY{p}{(}\PY{n+nx}{n}\PY{p}{)}\PY{p}{(}\PY{n+nx}{k}\PY{p}{)}\PY{p}{)}\PY{p}{(}\PY{n+nx}{LEQ}\PY{p}{(}\PY{n+nx}{k}\PY{p}{)}\PY{p}{(}\PY{n+nx}{n}\PY{p}{)}\PY{p}{)}
\end{Verbatim}


\begin{Verbatim}[commandchars=\\\{\}]
{\color{outcolor}Out[{\color{outcolor}70}]:} [Function: EQ]
\end{Verbatim}
            
    \begin{Verbatim}[commandchars=\\\{\}]
{\color{incolor}In [{\color{incolor}71}]:} \PY{n+nx}{EQ}\PY{p}{(}\PY{n+nx}{N0}\PY{p}{)}\PY{p}{(}\PY{n+nx}{N0}\PY{p}{)}
\end{Verbatim}


\begin{Verbatim}[commandchars=\\\{\}]
{\color{outcolor}Out[{\color{outcolor}71}]:} T / K
\end{Verbatim}
            
    \begin{Verbatim}[commandchars=\\\{\}]
{\color{incolor}In [{\color{incolor}72}]:} \PY{n+nx}{EQ}\PY{p}{(}\PY{n+nx}{N1}\PY{p}{)}\PY{p}{(}\PY{n+nx}{N0}\PY{p}{)}
\end{Verbatim}


\begin{Verbatim}[commandchars=\\\{\}]
{\color{outcolor}Out[{\color{outcolor}72}]:} F / KI
\end{Verbatim}
            
    $ \text{GT} = \lambda nk.\text{NOT}(\text{LEQ }nk) = \text{B1 NOT LEQ }
$

    \begin{Verbatim}[commandchars=\\\{\}]
{\color{incolor}In [{\color{incolor}73}]:} \PY{n+nx}{GT} \PY{o}{=} \PY{n+nx}{n} \PY{p}{=\PYZgt{}} \PY{n+nx}{k} \PY{p}{=\PYZgt{}} \PY{n+nx}{NOT}\PY{p}{(}\PY{n+nx}{LEQ}\PY{p}{(}\PY{n+nx}{n}\PY{p}{)}\PY{p}{(}\PY{n+nx}{k}\PY{p}{)}\PY{p}{)}
\end{Verbatim}


\begin{Verbatim}[commandchars=\\\{\}]
{\color{outcolor}Out[{\color{outcolor}73}]:} [Function: GT]
\end{Verbatim}
            
    \begin{Verbatim}[commandchars=\\\{\}]
{\color{incolor}In [{\color{incolor}74}]:} \PY{n+nx}{GT} \PY{o}{=} \PY{n+nx}{B1}\PY{p}{(}\PY{n+nx}{NOT}\PY{p}{)}\PY{p}{(}\PY{n+nx}{LEQ}\PY{p}{)}
\end{Verbatim}


\begin{Verbatim}[commandchars=\\\{\}]
{\color{outcolor}Out[{\color{outcolor}74}]:} [Function]
\end{Verbatim}
            
    \begin{Verbatim}[commandchars=\\\{\}]
{\color{incolor}In [{\color{incolor}75}]:} \PY{n+nx}{GT}\PY{p}{(}\PY{n+nx}{N1}\PY{p}{)}\PY{p}{(}\PY{n+nx}{N0}\PY{p}{)}\PY{p}{(}\PY{l+s+s1}{\PYZsq{}T\PYZsq{}}\PY{p}{)}\PY{p}{(}\PY{l+s+s1}{\PYZsq{}F\PYZsq{}}\PY{p}{)}
\end{Verbatim}


\begin{Verbatim}[commandchars=\\\{\}]
{\color{outcolor}Out[{\color{outcolor}75}]:} 'T'
\end{Verbatim}
            
    \begin{Verbatim}[commandchars=\\\{\}]
{\color{incolor}In [{\color{incolor}76}]:} \PY{n+nx}{GT}\PY{p}{(}\PY{n+nx}{N0}\PY{p}{)}\PY{p}{(}\PY{n+nx}{N0}\PY{p}{)}\PY{p}{(}\PY{l+s+s1}{\PYZsq{}T\PYZsq{}}\PY{p}{)}\PY{p}{(}\PY{l+s+s1}{\PYZsq{}F\PYZsq{}}\PY{p}{)}
\end{Verbatim}


\begin{Verbatim}[commandchars=\\\{\}]
{\color{outcolor}Out[{\color{outcolor}76}]:} 'F'
\end{Verbatim}
            

    % Add a bibliography block to the postdoc
    
    
    
    \end{document}
